%============================================================================
% APPENDIX A: API REFERENCE
%============================================================================
\section{API Reference}
\label{appendix:api-reference}

This appendix provides comprehensive API documentation for all FLOPY-NET services, including endpoint specifications, request/response schemas, authentication requirements, and usage examples.

\subsection{Policy Engine API}

The Policy Engine exposes RESTful APIs for policy management, compliance checking, and event monitoring.

\subsubsection{Base URL and Authentication}

\begin{itemize}
    \item \textbf{Base URL}: \texttt{http://policy-engine:5000}
    \item \textbf{Authentication}: None (internal service)
    \item \textbf{Content Type}: \texttt{application/json}
\end{itemize}

\subsubsection{Health Check Endpoint}

\begin{table}[H]
\centering
\caption{Policy Engine Health Check API}
\label{tab:policy-health-api}
\begin{tabularx}{\textwidth}{@{}lX@{}}
\toprule
\textbf{Method} & GET \\
\textbf{Endpoint} & \texttt{/health} \\
\textbf{Description} & Check service health and status \\
\textbf{Parameters} & None \\
\textbf{Response} & JSON object with health status \\
\bottomrule
\end{tabularx}
\end{table}

\begin{lstlisting}[style=jsoncode, caption=Policy Engine Health Check Response]
{
  "status": "healthy",
  "timestamp": "2025-01-15T10:30:00Z",
  "version": "2.0.0",
  "uptime": 86400,
  "policies_loaded": 15,
  "events_processed": 1543,
  "memory_usage": "245MB",
  "storage_backend": "sqlite"
}
\end{lstlisting}

\subsubsection{Policy Management Endpoints}

\begin{table}[H]
\centering
\caption{Policy Management API Endpoints}
\label{tab:policy-management-api}
\begin{tabularx}{\textwidth}{@{}llX@{}}
\toprule
\textbf{Method} & \textbf{Endpoint} & \textbf{Description} \\
\midrule
GET & \texttt{/policies} & Retrieve all active policies \\
POST & \texttt{/policies} & Create a new policy \\
GET & \texttt{/policies/\{id\}} & Retrieve specific policy by ID \\
PUT & \texttt{/policies/\{id\}} & Update existing policy \\
DELETE & \texttt{/policies/\{id\}} & Delete policy by ID \\
POST & \texttt{/policies/reload} & Reload policies from configuration file \\
\bottomrule
\end{tabularx}
\end{table}

\paragraph{Create Policy Example}

\begin{lstlisting}[style=jsoncode, caption=Create Policy Request]
POST /policies
Content-Type: application/json

{
  "id": "fl_client_validation",
  "category": "fl_performance",
  "priority": 100,
  "name": "FL Client Validation Policy",
  "description": "Validate FL client participation requirements",
  "conditions": {
    "client_type": "fl_client",
    "min_data_size": 1000,
    "max_staleness": 5
  },
  "actions": {
    "primary": "ALLOW",
    "on_violation": "REJECT",
    "notification": true
  },
  "metadata": {
    "created_by": "admin",
    "environment": "production"
  }
}
\end{lstlisting}

\begin{lstlisting}[style=jsoncode, caption=Create Policy Response]
{
  "status": "created",
  "policy_id": "fl_client_validation",
  "message": "Policy created successfully",
  "validation": {
    "syntax_valid": true,
    "conflicts": [],
    "warnings": []
  },
  "timestamp": "2025-01-15T10:35:00Z"
}
\end{lstlisting}

\subsubsection{Policy Compliance Check}

\begin{table}[H]
\centering
\caption{Policy Compliance Check API}
\label{tab:policy-check-api}
\begin{tabularx}{\textwidth}{@{}lX@{}}
\toprule
\textbf{Method} & POST \\
\textbf{Endpoint} & \texttt{/check} \\
\textbf{Description} & Perform policy compliance check \\
\textbf{Request Body} & JSON object with check parameters \\
\textbf{Response} & Policy decision and details \\
\bottomrule
\end{tabularx}
\end{table}

\begin{lstlisting}[style=jsoncode, caption=Policy Check Request]
POST /check
Content-Type: application/json

{
  "type": "client_participation",
  "client_id": "client_001",
  "client_ip": "192.168.100.101",
  "data_size": 5000,
  "last_update": "2025-01-15T10:30:00Z",
  "context": {
    "training_round": 5,
    "total_clients": 10,
    "active_clients": 8
  }
}
\end{lstlisting}

\begin{lstlisting}[style=jsoncode, caption=Policy Check Response]
{
  "decision": "ALLOW",
  "reason": "Client meets all participation requirements",
  "matched_policies": [
    {
      "policy_id": "fl_client_validation",
      "priority": 100,
      "decision": "ALLOW"
    }
  ],
  "violations": [],
  "recommendations": [],
  "timestamp": "2025-01-15T10:35:00Z",
  "processing_time_ms": 5
}
\end{lstlisting}

\subsection{Collector Service API}

The Collector Service provides APIs for metrics ingestion, querying, and system monitoring.

\subsubsection{Base Configuration}

\begin{itemize}
    \item \textbf{Base URL}: \texttt{http://collector:8000}
    \item \textbf{Authentication}: API Key (X-API-Key header)
    \item \textbf{Rate Limits}: 1000 requests/minute per API key
\end{itemize}

\subsubsection{Metrics Ingestion}

\begin{table}[H]
\centering
\caption{Metrics Ingestion API}
\label{tab:metrics-ingestion-api}
\begin{tabularx}{\textwidth}{@{}lX@{}}
\toprule
\textbf{Method} & POST \\
\textbf{Endpoint} & \texttt{/metrics} \\
\textbf{Description} & Submit metrics data for storage \\
\textbf{Headers} & \texttt{X-API-Key: <api\_key>} \\
\textbf{Request Body} & Array of metric objects \\
\bottomrule
\end{tabularx}
\end{table}

\begin{lstlisting}[style=jsoncode, caption=Metrics Ingestion Request]
POST /metrics
X-API-Key: flopynet-api-key-001
Content-Type: application/json

[
  {
    "source": "fl_server",
    "metric_name": "training_accuracy",
    "value": 0.85,
    "timestamp": 1642234567.123,
    "labels": {
      "round": "5",
      "model_type": "cnn",
      "dataset": "cifar10"
    },
    "metadata": {
      "client_count": 8,
      "aggregation_method": "fedavg"
    }
  },
  {
    "source": "fl_client_001",
    "metric_name": "local_training_time",
    "value": 45.2,
    "timestamp": 1642234567.123,
    "labels": {
      "client_id": "001",
      "round": "5"
    }
  }
]
\end{lstlisting}

\subsubsection{Metrics Query}

\begin{table}[H]
\centering
\caption{Metrics Query API}
\label{tab:metrics-query-api}
\begin{tabularx}{\textwidth}{@{}lX@{}}
\toprule
\textbf{Method} & GET \\
\textbf{Endpoint} & \texttt{/metrics} \\
\textbf{Description} & Query historical metrics data \\
\textbf{Query Parameters} & Various filters and options \\
\bottomrule
\end{tabularx}
\end{table}

\paragraph{Query Parameters}

\begin{table}[H]
\centering
\caption{Metrics Query Parameters}
\label{tab:metrics-query-params}
\begin{tabularx}{\textwidth}{@{}llX@{}}
\toprule
\textbf{Parameter} & \textbf{Type} & \textbf{Description} \\
\midrule
\texttt{source} & string & Filter by metric source \\
\texttt{metric\_name} & string & Filter by metric name \\
\texttt{start\_time} & timestamp & Start of time range \\
\texttt{end\_time} & timestamp & End of time range \\
\texttt{limit} & integer & Maximum number of results (default: 1000) \\
\texttt{labels} & string & Label-based filtering (key=value format) \\
\texttt{aggregation} & string & Aggregation type (avg, sum, min, max) \\
\texttt{window} & string & Time window for aggregation (1m, 5m, 1h, 1d) \\
\bottomrule
\end{tabularx}
\end{table}

\begin{lstlisting}[style=pythoncode, caption=Metrics Query Examples]
# Query FL server accuracy metrics for last hour
GET /metrics?source=fl_server&metric_name=training_accuracy&start_time=1642231000&end_time=1642234600

# Query aggregated client training times
GET /metrics?source=fl_client_*&metric_name=local_training_time&aggregation=avg&window=5m

# Query metrics with label filtering
GET /metrics?labels=round=5,model_type=cnn&limit=500
\end{lstlisting}

\subsection{Dashboard API}

The Dashboard backend provides APIs for the web interface and external integrations.

\subsubsection{System Overview}

\begin{table}[H]
\centering
\caption{Dashboard System Overview API}
\label{tab:dashboard-overview-api}
\begin{tabularx}{\textwidth}{@{}lX@{}}
\toprule
\textbf{Method} & GET \\
\textbf{Endpoint} & \texttt{/api/v1/overview} \\
\textbf{Description} & Get comprehensive system overview \\
\textbf{Response} & JSON object with system status \\
\bottomrule
\end{tabularx}
\end{table}

\begin{lstlisting}[style=jsoncode, caption=System Overview Response]
{
  "system_status": {
    "overall_health": "healthy",
    "active_services": 6,
    "total_services": 6,
    "uptime": 86400
  },
  "fl_training": {
    "active_rounds": 1,
    "total_rounds": 15,
    "active_clients": 8,
    "total_clients": 10,
    "current_accuracy": 0.87,
    "convergence_status": "improving"
  },
  "network_status": {
    "topology_health": "good",
    "average_latency": 45.2,
    "packet_loss_rate": 0.001,
    "bandwidth_utilization": 0.65
  },
  "policy_compliance": {
    "active_policies": 15,
    "violations_last_hour": 2,
    "compliance_score": 0.98
  },
  "resource_usage": {
    "cpu_usage": 0.45,
    "memory_usage": 0.67,
    "disk_usage": 0.23,
    "network_io": {
      "bytes_sent": 1048576000,
      "bytes_received": 2097152000
    }
  }
}
\end{lstlisting}

\subsection{FL Server API}

The Federated Learning Server provides APIs for training coordination and model management.

\subsubsection{Training Control}

\begin{table}[H]
\centering
\caption{FL Training Control APIs}
\label{tab:fl-training-api}
\begin{tabularx}{\textwidth}{@{}llX@{}}
\toprule
\textbf{Method} & \textbf{Endpoint} & \textbf{Description} \\
\midrule
POST & \texttt{/training/start} & Start federated learning training \\
POST & \texttt{/training/stop} & Stop current training session \\
GET & \texttt{/training/status} & Get current training status \\
GET & \texttt{/training/rounds} & Get training round history \\
POST & \texttt{/training/configure} & Update training configuration \\
\bottomrule
\end{tabularx}
\end{table}

\subsection{Error Handling}

All APIs follow consistent error handling patterns:

\begin{lstlisting}[style=jsoncode, caption=Standard Error Response Format]
{
  "error": {
    "code": "POLICY_VIOLATION",
    "message": "Client does not meet participation requirements",
    "details": {
      "failed_conditions": ["min_data_size"],
      "required_data_size": 1000,
      "actual_data_size": 500
    },
    "timestamp": "2025-01-15T10:35:00Z",
    "request_id": "req_123456789"
  }
}
\end{lstlisting}

\paragraph{HTTP Status Codes}

\begin{itemize}
    \item \textbf{200 OK}: Request successful
    \item \textbf{201 Created}: Resource created successfully
    \item \textbf{400 Bad Request}: Invalid request parameters
    \item \textbf{401 Unauthorized}: Authentication required
    \item \textbf{403 Forbidden}: Access denied by policy
    \item \textbf{404 Not Found}: Resource not found
    \item \textbf{429 Too Many Requests}: Rate limit exceeded
    \item \textbf{500 Internal Server Error}: Server error
    \item \textbf{503 Service Unavailable}: Service temporarily unavailable
\end{itemize}

\subsection{API Client Libraries}

FLOPY-NET provides client libraries in multiple languages:

\begin{itemize}
    \item \textbf{Python}: \texttt{pip install flopynet-client}
    \item \textbf{JavaScript/Node.js}: \texttt{npm install flopynet-client}
    \item \textbf{Go}: Available via Go modules
    \item \textbf{Java}: Maven/Gradle artifacts
\end{itemize}

\begin{lstlisting}[style=pythoncode, caption=Python Client Library Example]
from flopynet_client import PolicyEngineClient, CollectorClient

# Initialize clients
policy_client = PolicyEngineClient("http://policy-engine:5000")
collector_client = CollectorClient("http://collector:8000", api_key="your-api-key")

# Check policy
result = await policy_client.check_policy({
    "type": "client_participation",
    "client_id": "client_001",
    "data_size": 5000
})

# Submit metrics
await collector_client.submit_metrics([
    {
        "source": "my_component",
        "metric_name": "custom_metric",
        "value": 42.0,
        "timestamp": time.time()
    }
])
\end{lstlisting}

This API reference provides the foundation for integrating with and extending the FLOPY-NET platform. For complete API documentation with interactive examples, refer to the OpenAPI specifications available at each service's \texttt{/docs} endpoint.
