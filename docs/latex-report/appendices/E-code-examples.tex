%============================================================================
% APPENDIX E: CODE EXAMPLES
%============================================================================
\section{Code Examples}
\label{appendix:code-examples}

This appendix provides practical code examples for integrating with and extending FLOPY-NET components, based on the actual system architecture and implementation patterns found in the codebase.

\subsection{Policy Engine Integration}

\subsubsection{Custom Policy Implementation}

\begin{lstlisting}[language=python, caption=Custom Policy Definition]
from policy_engine.models import Policy, Condition, Action
from policy_engine.storage import get_policy_storage
from typing import Dict, Any

class CustomPolicyManager:
    """Custom policy management for FL scenarios"""
    
    def __init__(self):
        self.storage = get_policy_storage()
    
    def create_fl_client_policy(self, client_requirements: Dict[str, Any]) -> Policy:
        """Create a policy for FL client validation"""
        
        conditions = []
        
        # Add data size requirement
        if 'min_data_size' in client_requirements:
            conditions.append(Condition(
                field="data_size",
                operator=">=",
                value=client_requirements['min_data_size']
            ))
        
        # Add trust score requirement
        if 'min_trust_score' in client_requirements:
            conditions.append(Condition(
                field="trust_score",
                operator=">=",
                value=client_requirements['min_trust_score']
            ))
        
        # Add client type validation
        conditions.append(Condition(
            field="client_type",
            operator="==",
            value="fl_client"
        ))
        
        # Define actions
        actions = [
            Action(
                type="fl_participation",
                target="allow",
                parameters={"priority": "normal"}
            ),
            Action(
                type="logging",
                target="audit_log",
                parameters={"level": "info"}
            )
        ]
        
        policy = Policy(
            name=f"FL Client Policy - {client_requirements.get('scenario', 'default')}",
            description="Validate FL client participation requirements",
            conditions=conditions,
            actions=actions,
            priority=100,
            enabled=True,
            tags=["fl", "client_validation"]
        )
        
        return self.storage.create_policy(policy)
    
    def create_network_optimization_policy(self, latency_threshold: float) -> Policy:
        """Create a policy for network optimization based on latency"""
        
        policy = Policy(
            name="Network Latency Optimization",
            description=f"Optimize network when latency exceeds {latency_threshold}ms",
            conditions=[
                Condition(
                    field="network_latency",
                    operator=">",
                    value=latency_threshold
                )
            ],
            actions=[
                Action(
                    type="sdn",
                    target="prioritize_traffic",
                    parameters={
                        "priority": "high",
                        "qos_class": "expedited_forwarding"
                    }
                ),
                Action(
                    type="notification",
                    target="alert",
                    parameters={
                        "severity": "warning",
                        "message": f"High latency detected: {latency_threshold}ms"
                    }
                )
            ],
            priority=200,
            enabled=True,
            tags=["network", "optimization", "latency"]
        )
        
        return self.storage.create_policy(policy)

# Usage example
policy_manager = CustomPolicyManager()

# Create client validation policy
client_policy = policy_manager.create_fl_client_policy({
    'min_data_size': 1000,
    'min_trust_score': 0.7,
    'scenario': 'healthcare'
})

# Create network optimization policy
network_policy = policy_manager.create_network_optimization_policy(50.0)
\end{lstlisting}

\subsubsection{Policy Evaluation Integration}

\begin{lstlisting}[language=python, caption=Policy Evaluation Client]
import requests
import json
from typing import Dict, Any, List

class PolicyEvaluationClient:
    """Client for policy evaluation integration"""
    
    def __init__(self, policy_engine_url: str = "http://policy-engine:5000"):
        self.base_url = policy_engine_url
        self.session = requests.Session()
    
    def evaluate_fl_client_participation(self, client_data: Dict[str, Any]) -> Dict[str, Any]:
        """Evaluate if a client can participate in FL training"""
        
        evaluation_context = {
            "client_id": client_data.get("client_id"),
            "client_type": "fl_client",
            "data_size": client_data.get("data_size", 0),
            "trust_score": client_data.get("trust_score", 0.0),
            "last_participation": client_data.get("last_participation"),
            "computational_capability": client_data.get("cpu_cores", 1)
        }
        
        try:
            response = self.session.post(
                f"{self.base_url}/policies/evaluate",
                json={
                    "context": evaluation_context,
                    "policy_types": ["fl_client_validation"]
                }
            )
            
            if response.status_code == 200:
                result = response.json()
                return {
                    "allowed": result.get("decision") == "ALLOW",
                    "reason": result.get("reason"),
                    "matched_policies": result.get("matched_policies", []),
                    "violations": result.get("violations", [])
                }
            else:
                return {
                    "allowed": False,
                    "reason": f"Policy evaluation failed: {response.status_code}",
                    "error": response.text
                }
        
        except Exception as e:
            return {
                "allowed": False,
                "reason": f"Policy evaluation error: {str(e)}",
                "error": str(e)
            }
    
    def check_network_optimization_needed(self, network_metrics: Dict[str, float]) -> bool:
        """Check if network optimization policies should be triggered"""
        
        context = {
            "network_latency": network_metrics.get("latency", 0.0),
            "packet_loss": network_metrics.get("packet_loss", 0.0),
            "bandwidth_utilization": network_metrics.get("bandwidth_util", 0.0),
            "jitter": network_metrics.get("jitter", 0.0)
        }
        
        try:
            response = self.session.post(
                f"{self.base_url}/policies/evaluate",
                json={
                    "context": context,
                    "policy_types": ["network_optimization"]
                }
            )
            
            if response.status_code == 200:
                result = response.json()
                return result.get("decision") == "OPTIMIZE"
            
            return False
        
        except Exception as e:
            print(f"Network optimization check failed: {e}")
            return False

# Usage example
policy_client = PolicyEvaluationClient()

# Evaluate client participation
client_data = {
    "client_id": "client_001",
    "data_size": 5000,
    "trust_score": 0.85,
    "cpu_cores": 4
}

evaluation_result = policy_client.evaluate_fl_client_participation(client_data)
print(f"Client participation allowed: {evaluation_result['allowed']}")
print(f"Reason: {evaluation_result['reason']}")
\end{lstlisting}

\subsection{FL Server Integration}

\subsubsection{Custom FL Server Extension}

\begin{lstlisting}[language=python, caption=Extended FL Server Implementation]
import asyncio
import aiohttp
from typing import Dict, List, Any
import logging
import json

class ExtendedFLServer:
    """Extended FL Server with policy integration and monitoring"""
    
    def __init__(self, config: Dict[str, Any]):
        self.config = config
        self.clients = {}
        self.training_rounds = 0
        self.policy_client = PolicyEvaluationClient(
            config.get('policy_engine_url', 'http://policy-engine:5000')
        )
        self.collector_url = config.get('collector_url', 'http://collector:8000')
        self.logger = logging.getLogger(__name__)
    
    async def register_client(self, client_data: Dict[str, Any]) -> Dict[str, Any]:
        """Register a new FL client with policy validation"""
        
        client_id = client_data.get('client_id')
        
        # Evaluate client eligibility through policy engine
        evaluation = self.policy_client.evaluate_fl_client_participation(client_data)
        
        if not evaluation['allowed']:
            self.logger.warning(f"Client {client_id} registration rejected: {evaluation['reason']}")
            return {
                "status": "rejected",
                "reason": evaluation['reason'],
                "client_id": client_id
            }
        
        # Register client
        self.clients[client_id] = {
            **client_data,
            "status": "registered",
            "last_seen": asyncio.get_event_loop().time(),
            "training_rounds": 0
        }
        
        # Report registration to collector
        await self.report_client_event(client_id, "registered")
        
        self.logger.info(f"Client {client_id} registered successfully")
        return {
            "status": "registered",
            "client_id": client_id,
            "message": "Client registered successfully"
        }
    
    async def start_training_round(self) -> Dict[str, Any]:
        """Start a new training round with policy-compliant clients"""
        
        eligible_clients = []
        
        # Check each client's current eligibility
        for client_id, client_data in self.clients.items():
            if client_data['status'] == 'registered':
                # Re-evaluate eligibility for this training round
                evaluation = self.policy_client.evaluate_fl_client_participation(client_data)
                if evaluation['allowed']:
                    eligible_clients.append(client_id)
                else:
                    self.logger.info(f"Client {client_id} excluded from round: {evaluation['reason']}")
        
        if len(eligible_clients) < self.config.get('min_clients', 2):
            return {
                "status": "insufficient_clients",
                "eligible_clients": len(eligible_clients),
                "required_clients": self.config.get('min_clients', 2)
            }
        
        # Start training round
        self.training_rounds += 1
        
        # Notify eligible clients
        round_config = {
            "round_number": self.training_rounds,
            "learning_rate": self.config.get('learning_rate', 0.01),
            "batch_size": self.config.get('batch_size', 32),
            "local_epochs": self.config.get('local_epochs', 5)
        }
        
        # Report training round start to collector
        await self.report_training_event("round_started", {
            "round_number": self.training_rounds,
            "participating_clients": eligible_clients
        })
        
        return {
            "status": "started",
            "round_number": self.training_rounds,
            "participating_clients": eligible_clients,
            "round_config": round_config
        }
    
    async def report_client_event(self, client_id: str, event_type: str):
        """Report client events to the collector service"""
        
        event_data = {
            "source": "fl_server",
            "event_type": f"client_{event_type}",
            "client_id": client_id,
            "timestamp": asyncio.get_event_loop().time(),
            "round_number": self.training_rounds
        }
        
        try:
            async with aiohttp.ClientSession() as session:
                async with session.post(
                    f"{self.collector_url}/events",
                    json=event_data
                ) as response:
                    if response.status != 200:
                        self.logger.warning(f"Failed to report event to collector: {response.status}")
        except Exception as e:
            self.logger.error(f"Error reporting event to collector: {e}")
    
    async def report_training_event(self, event_type: str, event_data: Dict[str, Any]):
        """Report training events to the collector service"""
        
        training_event = {
            "source": "fl_server",
            "event_type": f"training_{event_type}",
            "timestamp": asyncio.get_event_loop().time(),
            **event_data
        }
        
        try:
            async with aiohttp.ClientSession() as session:
                async with session.post(
                    f"{self.collector_url}/events",
                    json=training_event
                ) as response:
                    if response.status != 200:
                        self.logger.warning(f"Failed to report training event: {response.status}")
        except Exception as e:
            self.logger.error(f"Error reporting training event: {e}")
    
    def get_server_status(self) -> Dict[str, Any]:
        """Get current server status"""
        return {
            "status": "running",
            "training_rounds": self.training_rounds,
            "registered_clients": len(self.clients),
            "active_clients": len([c for c in self.clients.values() if c['status'] == 'registered']),
            "config": {
                "min_clients": self.config.get('min_clients', 2),
                "max_clients": self.config.get('max_clients', 10)
            }
        }

# Usage example
fl_config = {
    "policy_engine_url": "http://policy-engine:5000",
    "collector_url": "http://collector:8000",
    "min_clients": 2,
    "max_clients": 10,
    "learning_rate": 0.01,
    "batch_size": 32,
    "local_epochs": 5
}

fl_server = ExtendedFLServer(fl_config)

# Example client registration
client_registration = {
    "client_id": "client_001",
    "data_size": 5000,
    "trust_score": 0.9,
    "computational_capability": 4
}

# This would be called in an async context
# result = await fl_server.register_client(client_registration)
\end{lstlisting}

\subsection{SDN Controller Integration}

\subsubsection{Custom Network Optimization}

\begin{lstlisting}[language=python, caption=SDN Controller Integration]
import requests
import json
from typing import Dict, List, Any
import logging

class FLNetworkOptimizer:
    """Network optimization controller for FL traffic"""
    
    def __init__(self, sdn_controller_url: str = "http://sdn-controller:8181"):
        self.sdn_url = sdn_controller_url
        self.logger = logging.getLogger(__name__)
        self.fl_servers = set()
        self.fl_clients = set()
    
    def register_fl_topology(self, servers: List[str], clients: List[str]):
        """Register FL servers and clients with SDN controller"""
        
        # Register FL servers
        for server_ip in servers:
            try:
                response = requests.post(
                    f"{self.sdn_url}/fl/register/server",
                    json={"server_ip": server_ip}
                )
                if response.status_code == 200:
                    self.fl_servers.add(server_ip)
                    self.logger.info(f"Registered FL server: {server_ip}")
                else:
                    self.logger.error(f"Failed to register server {server_ip}: {response.text}")
            except Exception as e:
                self.logger.error(f"Error registering server {server_ip}: {e}")
        
        # Register FL clients
        for client_ip in clients:
            try:
                response = requests.post(
                    f"{self.sdn_url}/fl/register/client",
                    json={"client_ip": client_ip}
                )
                if response.status_code == 200:
                    self.fl_clients.add(client_ip)
                    self.logger.info(f"Registered FL client: {client_ip}")
                else:
                    self.logger.error(f"Failed to register client {client_ip}: {response.text}")
            except Exception as e:
                self.logger.error(f"Error registering client {client_ip}: {e}")
    
    def optimize_fl_communication(self, server_ip: str, client_ips: List[str], priority: int = 200):
        """Optimize network paths for FL communication"""
        
        optimization_results = []
        
        for client_ip in client_ips:
            try:
                response = requests.post(
                    f"{self.sdn_url}/fl/prioritize",
                    json={
                        "src_ip": client_ip,
                        "dst_ip": server_ip,
                        "priority": priority,
                        "duration": 600  # 10 minutes
                    }
                )
                
                if response.status_code == 200:
                    result = response.json()
                    optimization_results.append({
                        "client_ip": client_ip,
                        "status": "optimized",
                        "details": result
                    })
                    self.logger.info(f"Optimized path: {client_ip} -> {server_ip}")
                else:
                    optimization_results.append({
                        "client_ip": client_ip,
                        "status": "failed",
                        "error": response.text
                    })
            
            except Exception as e:
                optimization_results.append({
                    "client_ip": client_ip,
                    "status": "error",
                    "error": str(e)
                })
                self.logger.error(f"Error optimizing path {client_ip} -> {server_ip}: {e}")
        
        return optimization_results
    
    def get_network_statistics(self) -> Dict[str, Any]:
        """Get current network statistics from SDN controller"""
        
        try:
            response = requests.get(f"{self.sdn_url}/fl/stats")
            if response.status_code == 200:
                return response.json()
            else:
                self.logger.error(f"Failed to get network stats: {response.status_code}")
                return {}
        except Exception as e:
            self.logger.error(f"Error getting network stats: {e}")
            return {}
    
    def monitor_fl_traffic(self) -> Dict[str, Any]:
        """Monitor FL-specific traffic patterns"""
        
        network_stats = self.get_network_statistics()
        
        # Extract FL-relevant metrics
        fl_metrics = {
            "fl_servers": list(self.fl_servers),
            "fl_clients": list(self.fl_clients),
            "total_flows": network_stats.get("total_flows", 0),
            "fl_prioritized_flows": network_stats.get("fl_priority_flows", 0),
            "average_latency": network_stats.get("average_latency", 0.0),
            "network_utilization": network_stats.get("bandwidth_utilization", 0.0)
        }
        
        return fl_metrics

# Usage example
network_optimizer = FLNetworkOptimizer()

# Register FL topology
fl_servers = ["192.168.100.10"]
fl_clients = ["192.168.100.101", "192.168.100.102"]

network_optimizer.register_fl_topology(fl_servers, fl_clients)

# Optimize communication paths
optimization_results = network_optimizer.optimize_fl_communication(
    server_ip="192.168.100.10",
    client_ips=fl_clients,
    priority=250  # High priority
)

# Monitor traffic
traffic_stats = network_optimizer.monitor_fl_traffic()
print(f"FL Traffic Statistics: {json.dumps(traffic_stats, indent=2)}")
\end{lstlisting}

\subsection{Dashboard Integration}

\subsubsection{Custom Dashboard Component}

\begin{lstlisting}[language=python, caption=Dashboard Backend Extension]
from fastapi import FastAPI, HTTPException
from fastapi.responses import JSONResponse
import aiohttp
import asyncio
from typing import Dict, Any, List
import logging

class DashboardDataAggregator:
    """Aggregates data from multiple FLOPY-NET services for dashboard display"""
    
    def __init__(self, service_urls: Dict[str, str]):
        self.service_urls = service_urls
        self.logger = logging.getLogger(__name__)
    
    async def get_system_overview(self) -> Dict[str, Any]:
        """Get comprehensive system overview"""
        
        # Gather data from all services concurrently
        tasks = [
            self._get_policy_engine_status(),
            self._get_fl_server_status(),
            self._get_collector_metrics(),
            self._get_network_status()
        ]
        
        results = await asyncio.gather(*tasks, return_exceptions=True)
        
        policy_status, fl_status, collector_metrics, network_status = results
        
        # Build comprehensive overview
        overview = {
            "timestamp": asyncio.get_event_loop().time(),
            "services": {
                "policy_engine": self._safe_extract(policy_status),
                "fl_server": self._safe_extract(fl_status),
                "collector": self._safe_extract(collector_metrics),
                "network": self._safe_extract(network_status)
            },
            "system_health": self._assess_system_health(results)
        }
        
        return overview
    
    async def _get_policy_engine_status(self) -> Dict[str, Any]:
        """Get policy engine status"""
        try:
            async with aiohttp.ClientSession() as session:
                async with session.get(f"{self.service_urls['policy_engine']}/health") as response:
                    if response.status == 200:
                        data = await response.json()
                        # Get additional policy statistics
                        async with session.get(f"{self.service_urls['policy_engine']}/policies") as policies_response:
                            if policies_response.status == 200:
                                policies = await policies_response.json()
                                data['total_policies'] = len(policies)
                                data['enabled_policies'] = len([p for p in policies if p.get('enabled', False)])
                        return data
                    else:
                        return {"status": "error", "code": response.status}
        except Exception as e:
            return {"status": "error", "message": str(e)}
    
    async def _get_fl_server_status(self) -> Dict[str, Any]:
        """Get FL server status"""
        try:
            async with aiohttp.ClientSession() as session:
                # Get server status
                async with session.get(f"{self.service_urls['fl_server']}/status") as response:
                    if response.status == 200:
                        status_data = await response.json()
                        
                        # Get client information
                        async with session.get(f"{self.service_urls['fl_server']}/clients") as clients_response:
                            if clients_response.status == 200:
                                clients_data = await clients_response.json()
                                status_data['clients'] = clients_data
                        
                        # Get training metrics
                        async with session.get(f"{self.service_urls['fl_server']}/metrics") as metrics_response:
                            if metrics_response.status == 200:
                                metrics_data = await metrics_response.json()
                                status_data['metrics'] = metrics_data
                        
                        return status_data
                    else:
                        return {"status": "error", "code": response.status}
        except Exception as e:
            return {"status": "error", "message": str(e)}
    
    async def _get_collector_metrics(self) -> Dict[str, Any]:
        """Get collector service metrics"""
        try:
            async with aiohttp.ClientSession() as session:
                async with session.get(f"{self.service_urls['collector']}/metrics/summary") as response:
                    if response.status == 200:
                        return await response.json()
                    else:
                        return {"status": "error", "code": response.status}
        except Exception as e:
            return {"status": "error", "message": str(e)}
    
    async def _get_network_status(self) -> Dict[str, Any]:
        """Get network and SDN controller status"""
        try:
            async with aiohttp.ClientSession() as session:
                async with session.get(f"{self.service_urls['sdn_controller']}/fl/stats") as response:
                    if response.status == 200:
                        return await response.json()
                    else:
                        return {"status": "error", "code": response.status}
        except Exception as e:
            return {"status": "error", "message": str(e)}
    
    def _safe_extract(self, data: Any) -> Dict[str, Any]:
        """Safely extract data from service responses"""
        if isinstance(data, Exception):
            return {"status": "error", "message": str(data)}
        elif isinstance(data, dict):
            return data
        else:
            return {"status": "unknown", "data": str(data)}
    
    def _assess_system_health(self, service_results: List[Any]) -> Dict[str, Any]:
        """Assess overall system health based on service responses"""
        
        healthy_services = 0
        total_services = len(service_results)
        
        for result in service_results:
            if isinstance(result, dict) and result.get("status") != "error":
                healthy_services += 1
        
        health_percentage = (healthy_services / total_services) * 100
        
        if health_percentage >= 90:
            health_status = "excellent"
        elif health_percentage >= 75:
            health_status = "good"
        elif health_percentage >= 50:
            health_status = "degraded"
        else:
            health_status = "critical"
        
        return {
            "status": health_status,
            "healthy_services": healthy_services,
            "total_services": total_services,
            "health_percentage": health_percentage
        }

# FastAPI integration
app = FastAPI(title="FLOPY-NET Dashboard API")

# Initialize data aggregator
service_urls = {
    "policy_engine": "http://policy-engine:5000",
    "fl_server": "http://fl-server:8080",
    "collector": "http://collector:8000",
    "sdn_controller": "http://sdn-controller:8181"
}

data_aggregator = DashboardDataAggregator(service_urls)

@app.get("/api/system/overview")
async def get_system_overview():
    """Get comprehensive system overview"""
    try:
        overview = await data_aggregator.get_system_overview()
        return JSONResponse(content=overview)
    except Exception as e:
        raise HTTPException(status_code=500, detail=str(e))

@app.get("/api/fl/training/status")
async def get_fl_training_status():
    """Get current FL training status"""
    try:
        fl_status = await data_aggregator._get_fl_server_status()
        return JSONResponse(content=fl_status)
    except Exception as e:
        raise HTTPException(status_code=500, detail=str(e))

# Usage example would be running the FastAPI app
# uvicorn extended_dashboard:app --host 0.0.0.0 --port 8001
\end{lstlisting}

\subsection{Collector Service Integration}

\subsubsection{Custom Metrics Collection}

\begin{lstlisting}[language=python, caption=Custom Metrics Collector]
import asyncio
import aiohttp
import time
import json
from typing import Dict, Any, List
import logging

class CustomMetricsCollector:
    """Custom metrics collector for FL-specific metrics"""
    
    def __init__(self, collector_url: str = "http://collector:8000"):
        self.collector_url = collector_url
        self.logger = logging.getLogger(__name__)
    
    async def collect_fl_training_metrics(self, fl_server_url: str, round_number: int):
        """Collect FL training metrics and submit to collector"""
        
        try:
            async with aiohttp.ClientSession() as session:
                # Get training metrics from FL server
                async with session.get(f"{fl_server_url}/metrics") as response:
                    if response.status == 200:
                        metrics_data = await response.json()
                        
                        # Transform metrics for collector
                        collector_metrics = []
                        
                        # Training round metrics
                        if 'accuracy' in metrics_data:
                            for i, accuracy in enumerate(metrics_data['accuracy']):
                                collector_metrics.append({
                                    "source": "fl_training",
                                    "metric_name": "accuracy",
                                    "value": accuracy,
                                    "timestamp": time.time(),
                                    "metadata": {
                                        "round": i + 1,
                                        "training_session": round_number
                                    }
                                })
                        
                        if 'loss' in metrics_data:
                            for i, loss in enumerate(metrics_data['loss']):
                                collector_metrics.append({
                                    "source": "fl_training",
                                    "metric_name": "loss",
                                    "value": loss,
                                    "timestamp": time.time(),
                                    "metadata": {
                                        "round": i + 1,
                                        "training_session": round_number
                                    }
                                })
                        
                        # Submit metrics to collector
                        await self.submit_metrics(collector_metrics)
                        
                        self.logger.info(f"Collected {len(collector_metrics)} FL training metrics")
                    
                    else:
                        self.logger.error(f"Failed to get FL metrics: {response.status}")
        
        except Exception as e:
            self.logger.error(f"Error collecting FL training metrics: {e}")
    
    async def collect_network_performance_metrics(self, sdn_controller_url: str):
        """Collect network performance metrics from SDN controller"""
        
        try:
            async with aiohttp.ClientSession() as session:
                async with session.get(f"{sdn_controller_url}/fl/stats") as response:
                    if response.status == 200:
                        network_stats = await response.json()
                        
                        # Transform network metrics
                        collector_metrics = []
                        
                        # Network latency
                        if 'average_latency' in network_stats:
                            collector_metrics.append({
                                "source": "network",
                                "metric_name": "average_latency",
                                "value": network_stats['average_latency'],
                                "timestamp": time.time(),
                                "metadata": {"unit": "ms"}
                            })
                        
                        # Bandwidth utilization
                        if 'bandwidth_utilization' in network_stats:
                            collector_metrics.append({
                                "source": "network",
                                "metric_name": "bandwidth_utilization",
                                "value": network_stats['bandwidth_utilization'],
                                "timestamp": time.time(),
                                "metadata": {"unit": "percentage"}
                            })
                        
                        # Flow statistics
                        if 'total_flows' in network_stats:
                            collector_metrics.append({
                                "source": "network",
                                "metric_name": "total_flows",
                                "value": network_stats['total_flows'],
                                "timestamp": time.time(),
                                "metadata": {"type": "count"}
                            })
                        
                        # Submit metrics
                        await self.submit_metrics(collector_metrics)
                        
                        self.logger.info(f"Collected {len(collector_metrics)} network metrics")
        
        except Exception as e:
            self.logger.error(f"Error collecting network metrics: {e}")
    
    async def collect_policy_compliance_metrics(self, policy_engine_url: str):
        """Collect policy compliance metrics"""
        
        try:
            async with aiohttp.ClientSession() as session:
                # Get policy statistics
                async with session.get(f"{policy_engine_url}/policies/stats") as response:
                    if response.status == 200:
                        policy_stats = await response.json()
                        
                        collector_metrics = []
                        
                        # Policy decision metrics
                        for decision_type, count in policy_stats.get('decisions', {}).items():
                            collector_metrics.append({
                                "source": "policy_engine",
                                "metric_name": f"policy_decisions_{decision_type}",
                                "value": count,
                                "timestamp": time.time(),
                                "metadata": {"decision_type": decision_type}
                            })
                        
                        # Policy evaluation time
                        if 'average_evaluation_time' in policy_stats:
                            collector_metrics.append({
                                "source": "policy_engine",
                                "metric_name": "policy_evaluation_time",
                                "value": policy_stats['average_evaluation_time'],
                                "timestamp": time.time(),
                                "metadata": {"unit": "ms"}
                            })
                        
                        await self.submit_metrics(collector_metrics)
                        
                        self.logger.info(f"Collected {len(collector_metrics)} policy metrics")
        
        except Exception as e:
            self.logger.error(f"Error collecting policy metrics: {e}")
    
    async def submit_metrics(self, metrics: List[Dict[str, Any]]):
        """Submit metrics to collector service"""
        
        try:
            async with aiohttp.ClientSession() as session:
                async with session.post(
                    f"{self.collector_url}/metrics",
                    json={"metrics": metrics}
                ) as response:
                    if response.status == 200:
                        self.logger.debug(f"Successfully submitted {len(metrics)} metrics")
                    else:
                        self.logger.error(f"Failed to submit metrics: {response.status}")
        
        except Exception as e:
            self.logger.error(f"Error submitting metrics: {e}")
    
    async def run_collection_cycle(self, service_urls: Dict[str, str], interval: int = 30):
        """Run continuous metrics collection cycle"""
        
        self.logger.info(f"Starting metrics collection cycle (interval: {interval}s)")
        
        while True:
            try:
                # Collect metrics from all services
                await asyncio.gather(
                    self.collect_fl_training_metrics(service_urls['fl_server'], int(time.time())),
                    self.collect_network_performance_metrics(service_urls['sdn_controller']),
                    self.collect_policy_compliance_metrics(service_urls['policy_engine']),
                    return_exceptions=True
                )
                
                self.logger.debug("Completed metrics collection cycle")
                
            except Exception as e:
                self.logger.error(f"Error in metrics collection cycle: {e}")
            
            # Wait for next collection interval
            await asyncio.sleep(interval)

# Usage example
async def main():
    collector = CustomMetricsCollector()
    
    service_urls = {
        'fl_server': 'http://fl-server:8080',
        'sdn_controller': 'http://sdn-controller:8181',
        'policy_engine': 'http://policy-engine:5000'
    }
    
    # Run continuous collection
    await collector.run_collection_cycle(service_urls, interval=30)

# Run with: asyncio.run(main())
\end{lstlisting}

These code examples demonstrate practical integration patterns with FLOPY-NET components, providing templates for extending the system's functionality while maintaining compatibility with the existing architecture.
