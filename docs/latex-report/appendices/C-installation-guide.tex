%============================================================================
% APPENDIX C: INSTALLATION GUIDE
%============================================================================
\section{Installation Guide}
\label{appendix:installation-guide}

This appendix provides comprehensive installation instructions for deploying FLOPY-NET in various environments, from development setups to production deployments.

\subsection{System Requirements}

\subsubsection{Hardware Requirements}

\begin{table}[H]
\centering
\caption{Hardware Requirements by Deployment Type}
\label{tab:hardware-requirements}
\begin{tabularx}{\textwidth}{@{}lXXX@{}}
\toprule
\textbf{Component} & \textbf{Development} & \textbf{Testing} & \textbf{Production} \\
\midrule
CPU & 4 cores & 8 cores & 16+ cores \\
RAM & 8 GB & 16 GB & 32+ GB \\
Storage & 50 GB SSD & 200 GB SSD & 1TB+ SSD \\
Network & 1 Gbps & 1 Gbps & 10+ Gbps \\
\bottomrule
\end{tabularx}
\end{table}

\subsubsection{Software Requirements}

\begin{itemize}
    \item \textbf{Operating System}: Linux (Ubuntu 20.04+, CentOS 8+) or Windows 10+ with WSL2
    \item \textbf{Docker}: Version 20.10.0 or higher
    \item \textbf{Docker Compose}: Version 2.0.0 or higher
    \item \textbf{Python}: Version 3.8+ (for development)
    \item \textbf{Node.js}: Version 16+ (for dashboard development)
    \item \textbf{GNS3}: Version 2.2+ (for network simulation)
\end{itemize}

\subsection{Quick Start Installation}

\subsubsection{Clone Repository}

\begin{lstlisting}[style=bashcode, caption=Repository Setup]
# Clone the FLOPY-NET repository
git clone https://github.com/your-org/flopy-net.git
cd flopy-net

# Verify repository structure
ls -la
# Should show: dashboard/, docker/, config/, scripts/, docker-compose.yml
\end{lstlisting}

\subsubsection{Environment Setup}

\begin{lstlisting}[style=bashcode, caption=Environment Configuration]
# Copy environment template
cp .env.example .env

# Edit configuration (optional)
nano .env

# Verify Docker installation
docker --version
docker-compose --version
\end{lstlisting}

\subsubsection{Start Services}

\begin{lstlisting}[style=bashcode, caption=Service Startup]
# Start all services
docker-compose up -d

# Verify services are running
docker-compose ps

# Check service logs
docker-compose logs -f policy-engine
docker-compose logs -f fl-server
\end{lstlisting}

\subsection{Detailed Installation Steps}

\subsubsection{Docker Environment Setup}

\begin{lstlisting}[style=bashcode, caption=Docker Installation (Ubuntu)]
# Update package index
sudo apt update

# Install Docker
sudo apt install -y docker.io docker-compose

# Add user to docker group (requires logout/login)
sudo usermod -aG docker $USER

# Start Docker service
sudo systemctl start docker
sudo systemctl enable docker

# Verify installation
docker run hello-world
\end{lstlisting}

\subsubsection{Container Registry Access}

The system uses pre-built images from Docker Hub under the \texttt{abdulmelink} namespace:

\begin{lstlisting}[style=bashcode, caption=Container Image Verification]
# Verify image access
docker pull abdulmelink/flopynet-policy-engine:v1.0.0-alpha.8
docker pull abdulmelink/flopynet-server:v1.0.0-alpha.8
docker pull abdulmelink/flopynet-client:v1.0.0-alpha.8
docker pull abdulmelink/flopynet-sdn-controller:v1.0.0-alpha.8

# List downloaded images
docker images | grep flopynet
\end{lstlisting}

\subsubsection{Network Configuration}

The system uses static IP configuration within the Docker network:

\begin{lstlisting}[style=dockercode, caption=Network Configuration]
# Network configuration in docker-compose.yml
networks:
  flopynet_network:
    driver: bridge
    ipam:
      driver: default
      config:
        - subnet: 192.168.100.0/24
          gateway: 192.168.100.1

# Service IP assignments:
# - Policy Engine: 192.168.100.20
# - FL Server: 192.168.100.10
# - FL Clients: 192.168.100.101-102
# - SDN Controller: 192.168.100.41
# - Collector: 192.168.100.40
\end{lstlisting}

\subsection{Service-Specific Configuration}

\subsubsection{Policy Engine Configuration}

\begin{lstlisting}[style=bashcode, caption=Policy Engine Setup]
# Create policy configuration directory
mkdir -p config/policies

# Create basic policy configuration
cat > config/policies/policy_config.json << EOF
{
  "policies": [
    {
      "id": "fl_client_basic",
      "name": "Basic FL Client Policy",
      "enabled": true,
      "conditions": [
        {
          "field": "client_type",
          "operator": "==",
          "value": "fl_client"
        }
      ],
      "actions": [
        {
          "type": "allow",
          "target": "fl_participation"
        }
      ]
    }
  ]
}
EOF
\end{lstlisting}

\subsubsection{FL Server Configuration}

\begin{lstlisting}[style=bashcode, caption=FL Server Setup]
# Create FL server configuration
mkdir -p config/fl_server

# Create server configuration
cat > config/fl_server/server_config.json << EOF
{
  "server": {
    "host": "0.0.0.0",
    "port": 8080,
    "min_clients": 2,
    "max_clients": 10,
    "rounds": 10
  },
  "model": {
    "type": "simple_nn",
    "learning_rate": 0.01,
    "batch_size": 32
  }
}
EOF
\end{lstlisting}

\subsubsection{Dashboard Configuration}

\begin{lstlisting}[style=bashcode, caption=Dashboard Setup]
# Dashboard backend configuration
mkdir -p config/dashboard

# Create dashboard configuration
cat > config/dashboard/dashboard_config.json << EOF
{
  "dashboard": {
    "host": "0.0.0.0",
    "port": 8085,
    "refresh_interval": 5
  },
  "services": {
    "policy_engine_url": "http://policy-engine:5000",
    "fl_server_url": "http://fl-server:8080",
    "collector_url": "http://collector:8000"
  }
}
EOF
\end{lstlisting}

\subsection{Verification and Testing}

\subsubsection{Service Health Checks}

\begin{lstlisting}[style=bashcode, caption=Health Check Verification]
# Check all services are healthy
docker-compose ps

# Test individual service endpoints
curl http://localhost:5000/health  # Policy Engine
curl http://localhost:8080/status  # FL Server
curl http://localhost:8000/health  # Collector

# Test dashboard access
curl http://localhost:8085/
\end{lstlisting}

\subsubsection{Functional Testing}

\begin{lstlisting}[style=bashcode, caption=Basic Functionality Test]
# Test policy engine
curl -X POST http://localhost:5000/policies \
  -H "Content-Type: application/json" \
  -d '{"name": "test", "description": "test policy", "rules": []}'

# Test FL server
curl -X POST http://localhost:8080/start

# Test dashboard API
curl http://localhost:8001/api/system/status
\end{lstlisting}

\subsection{Troubleshooting}

\subsubsection{Common Issues}

\begin{table}[H]
\centering
\caption{Common Installation Issues and Solutions}
\label{tab:troubleshooting}
\begin{tabularx}{\textwidth}{@{}lX@{}}
\toprule
\textbf{Issue} & \textbf{Solution} \\
\midrule
Port conflicts & Change port mappings in docker-compose.yml \\
Permission denied & Add user to docker group, restart session \\
Images not found & Verify Docker Hub access, check image tags \\
Network issues & Check firewall settings, verify network configuration \\
Service startup failures & Check logs with docker-compose logs SERVICE \\
\bottomrule
\end{tabularx}
\end{table}

\subsubsection{Log Analysis}

\begin{lstlisting}[style=bashcode, caption=Log Analysis Commands]
# View all service logs
docker-compose logs

# View specific service logs
docker-compose logs -f policy-engine
docker-compose logs -f fl-server
docker-compose logs -f collector

# View logs with timestamps
docker-compose logs -t policy-engine

# Filter logs by level
docker-compose logs policy-engine | grep ERROR
\end{lstlisting}

\subsection{Advanced Configuration}

\subsubsection{Production Deployment}

\begin{lstlisting}[style=dockercode, caption=Production Configuration]
# Production docker-compose.override.yml
version: '3.8'

services:
  policy-engine:
    environment:
      - LOG_LEVEL=INFO
      - POLICY_ENGINE_DEBUG=false
    deploy:
      resources:
        limits:
          cpus: '2.0'
          memory: 4G
        reservations:
          cpus: '1.0'
          memory: 2G

  fl-server:
    environment:
      - LOG_LEVEL=INFO
      - FL_SERVER_DEBUG=false
    deploy:
      resources:
        limits:
          cpus: '4.0'
          memory: 8G
\end{lstlisting}

\subsubsection{Scaling Configuration}

\begin{lstlisting}[style=bashcode, caption=Horizontal Scaling]
# Scale FL clients
docker-compose up -d --scale fl-client=5

# Scale with environment variables
CLIENT_COUNT=10 docker-compose up -d --scale fl-client=${CLIENT_COUNT}

# Verify scaling
docker-compose ps | grep fl-client
\end{lstlisting}

\subsection{Security Configuration}

\subsubsection{SSL/TLS Setup}

\begin{lstlisting}[style=bashcode, caption=SSL Certificate Setup]
# Create SSL certificates directory
mkdir -p config/ssl

# Generate self-signed certificates (development only)
openssl req -x509 -newkey rsa:4096 -keyout config/ssl/key.pem \
  -out config/ssl/cert.pem -days 365 -nodes \
  -subj "/C=US/ST=State/L=City/O=Org/CN=localhost"

# Set proper permissions
chmod 600 config/ssl/key.pem
chmod 644 config/ssl/cert.pem
\end{lstlisting}

\subsubsection{Environment Security}

\begin{lstlisting}[style=bashcode, caption=Security Hardening]
# Create secure environment file
cat > .env.secure << EOF
# Security settings
POLICY_ENGINE_DEBUG=false
FL_SERVER_DEBUG=false
LOG_LEVEL=INFO

# Database security
DB_PASSWORD=$(openssl rand -base64 32)
REDIS_PASSWORD=$(openssl rand -base64 32)

# API security
API_KEY=$(openssl rand -base64 32)
JWT_SECRET=$(openssl rand -base64 32)
EOF

# Secure the environment file
chmod 600 .env.secure
\end{lstlisting}

This installation guide provides comprehensive instructions for deploying FLOPY-NET across different environments while maintaining security best practices and proper configuration management.
