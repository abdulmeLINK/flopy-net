%============================================================================
% APPENDIX D: TROUBLESHOOTING GUIDE
%============================================================================
\section{Troubleshooting Guide}
\label{appendix:troubleshooting}

This appendix provides comprehensive troubleshooting guidance for common issues encountered when deploying and operating FLOPY-NET, organized by component and symptom type.

\subsection{General System Issues}

\subsubsection{Docker and Container Issues}

\begin{table}[H]
\centering
\caption{Docker-Related Issues and Solutions}
\label{tab:docker-issues}
\begin{tabularx}{\textwidth}{@{}lXX@{}}
\toprule
\textbf{Issue} & \textbf{Symptoms} & \textbf{Solution} \\
\midrule
Container won't start & Exit code 125, permission errors & Check Docker daemon, user permissions \\
Port conflicts & "Port already in use" errors & Change port mappings, check running services \\
Image pull failures & Network timeouts, authentication errors & Verify Docker Hub access, check credentials \\
Memory issues & OOMKilled status, container crashes & Increase memory limits, check resource usage \\
Network isolation & Services can't communicate & Verify network configuration, check firewall \\
\bottomrule
\end{tabularx}
\end{table}

\begin{lstlisting}[style=bashcode, caption=Docker Troubleshooting Commands]
# Check Docker daemon status
sudo systemctl status docker

# View container resource usage
docker stats

# Inspect container configuration
docker inspect CONTAINER_NAME

# Check container logs
docker logs CONTAINER_NAME --tail 100

# Check network connectivity
docker network ls
docker network inspect flopynet_network
\end{lstlisting}

\subsubsection{Service Discovery Issues}

\begin{lstlisting}[style=bashcode, caption=Service Discovery Debugging]
# Test service connectivity within Docker network
docker exec policy-engine ping fl-server
docker exec fl-server ping collector

# Check DNS resolution
docker exec policy-engine nslookup fl-server
docker exec policy-engine cat /etc/hosts

# Verify service endpoints
docker exec policy-engine curl http://fl-server:8080/status
docker exec fl-server curl http://policy-engine:5000/health
\end{lstlisting}

\subsection{Policy Engine Issues}

\subsubsection{Policy Engine Startup Problems}

\begin{table}[H]
\centering
\caption{Policy Engine Troubleshooting}
\label{tab:policy-engine-issues}
\begin{tabularx}{\textwidth}{@{}lXX@{}}
\toprule
\textbf{Issue} & \textbf{Symptoms} & \textbf{Solution} \\
\midrule
Configuration errors & Startup failures, JSON parse errors & Validate policy\_config.json syntax \\
Database issues & SQLite errors, permission denied & Check database file permissions \\
Port binding failures & Address already in use & Change POLICY\_ENGINE\_PORT environment variable \\
Policy loading errors & Invalid policy warnings & Validate policy syntax and structure \\
\bottomrule
\end{tabularx}
\end{table}

\begin{lstlisting}[style=bashcode, caption=Policy Engine Diagnostics]
# Check policy engine logs
docker-compose logs policy-engine

# Validate policy configuration
docker exec policy-engine python -m json.tool /app/config/policies/policy_config.json

# Test policy engine API
curl -v http://localhost:5000/health
curl -v http://localhost:5000/policies

# Check database file
docker exec policy-engine ls -la /data/
docker exec policy-engine sqlite3 /data/policies.db ".tables"
\end{lstlisting}

\subsubsection{Policy Evaluation Issues}

\begin{lstlisting}[style=pythoncode, caption=Policy Debugging Script]
#!/usr/bin/env python3
import requests
import json

def debug_policy_engine():
    """Debug policy engine functionality"""
    base_url = "http://localhost:5000"
    
    # Test health endpoint
    try:
        response = requests.get(f"{base_url}/health")
        print(f"Health check: {response.status_code}")
        print(f"Response: {response.json()}")
    except Exception as e:
        print(f"Health check failed: {e}")
    
    # Test policy listing
    try:
        response = requests.get(f"{base_url}/policies")
        print(f"Policies: {response.status_code}")
        print(f"Count: {len(response.json())}")
    except Exception as e:
        print(f"Policy listing failed: {e}")
    
    # Test policy creation
    test_policy = {
        "name": "test_policy",
        "description": "Test policy for debugging",
        "rules": [{"field": "test", "operator": "==", "value": "debug"}]
    }
    
    try:
        response = requests.post(f"{base_url}/policies", json=test_policy)
        print(f"Policy creation: {response.status_code}")
        if response.status_code == 201:
            print("Policy created successfully")
        else:
            print(f"Error: {response.text}")
    except Exception as e:
        print(f"Policy creation failed: {e}")

if __name__ == "__main__":
    debug_policy_engine()
\end{lstlisting}

\subsection{FL Server Issues}

\subsubsection{FL Server Connectivity Problems}

\begin{table}[H]
\centering
\caption{FL Server Troubleshooting}
\label{tab:fl-server-issues}
\begin{tabularx}{\textwidth}{@{}lXX@{}}
\toprule
\textbf{Issue} & \textbf{Symptoms} & \textbf{Solution} \\
\midrule
Client connection failures & Timeout errors, connection refused & Check network configuration, firewall rules \\
Training not starting & Server idle, no client participation & Verify minimum client threshold \\
Policy engine integration & Authorization failures & Check policy engine connectivity \\
Resource exhaustion & High memory/CPU usage & Monitor resource usage, adjust limits \\
\bottomrule
\end{tabularx}
\end{table}

\begin{lstlisting}[style=bashcode, caption=FL Server Diagnostics]
# Check FL server status
curl http://localhost:8080/status

# Monitor FL server logs
docker-compose logs -f fl-server

# Test FL server endpoints
curl -X POST http://localhost:8080/start
curl http://localhost:8080/clients
curl http://localhost:8080/metrics

# Check resource usage
docker stats fl-server
\end{lstlisting}

\subsubsection{Client Registration Issues}

\begin{lstlisting}[style=bashcode, caption=Client Registration Debugging]
# Check client logs
docker-compose logs fl-client-1
docker-compose logs fl-client-2

# Test client registration manually
curl -X POST http://localhost:8080/register \
  -H "Content-Type: application/json" \
  -d '{"client_id": "debug_client", "capabilities": ["basic"]}'

# Verify client list
curl http://localhost:8080/clients | jq .
\end{lstlisting}

\subsection{Dashboard Issues}

\subsubsection{Dashboard Connectivity Problems}

\begin{table}[H]
\centering
\caption{Dashboard Troubleshooting}
\label{tab:dashboard-issues}
\begin{tabularx}{\textwidth}{@{}lXX@{}}
\toprule
\textbf{Issue} & \textbf{Symptoms} & \textbf{Solution} \\
\midrule
Frontend not loading & Blank page, 404 errors & Check dashboard-frontend service status \\
API connection failures & Empty data, connection errors & Verify dashboard-backend connectivity \\
Real-time updates not working & Stale data, no refreshing & Check WebSocket connections \\
Service integration issues & Missing metrics, partial data & Verify service endpoint configuration \\
\bottomrule
\end{tabularx}
\end{table}

\begin{lstlisting}[style=bashcode, caption=Dashboard Diagnostics]
# Check dashboard services
docker-compose ps | grep dashboard

# Test dashboard backend API
curl http://localhost:8001/api/system/status
curl http://localhost:8001/api/fl/status
curl http://localhost:8001/api/policy/status

# Check frontend access
curl -I http://localhost:8085/

# Monitor dashboard logs
docker-compose logs -f dashboard-backend
docker-compose logs -f dashboard-frontend
\end{lstlisting}

\subsection{Network and SDN Issues}

\subsubsection{SDN Controller Problems}

\begin{table}[H]
\centering
\caption{SDN Controller Troubleshooting}
\label{tab:sdn-issues}
\begin{tabularx}{\textwidth}{@{}lXX@{}}
\toprule
\textbf{Issue} & \textbf{Symptoms} & \textbf{Solution} \\
\midrule
Controller startup failures & Ryu application errors & Check Ryu configuration, Python dependencies \\
Switch connectivity issues & No switches detected & Verify GNS3 configuration, OpenFlow settings \\
Flow installation failures & Traffic not prioritized & Check OpenFlow version compatibility \\
API endpoint errors & REST API not responding & Verify Ryu REST API configuration \\
\bottomrule
\end{tabularx}
\end{table}

\begin{lstlisting}[style=bashcode, caption=SDN Controller Diagnostics]
# Check SDN controller logs
docker-compose logs sdn-controller

# Test SDN controller API
curl http://localhost:8181/stats/switches
curl http://localhost:8181/stats/flows/1

# Test FL-specific endpoints
curl -X POST http://localhost:8181/fl/register/server \
  -H "Content-Type: application/json" \
  -d '{"server_ip": "192.168.100.10"}'

# Check Ryu application status
docker exec sdn-controller ps aux | grep ryu
\end{lstlisting}

\subsubsection{GNS3 Integration Issues}

\begin{lstlisting}[style=bashcode, caption=GNS3 Troubleshooting]
# Check GNS3 server status (if running separately)
curl http://localhost:3080/v2/version

# Test GNS3 project access
curl http://localhost:3080/v2/projects

# Check virtual network interfaces
docker exec sdn-controller ip addr show

# Test OpenFlow connectivity
docker exec sdn-controller nc -zv gns3-server 6633
\end{lstlisting}

\subsection{Performance Issues}

\subsubsection{Resource Utilization Problems}

\begin{lstlisting}[style=bashcode, caption=Performance Diagnostics]
# Monitor overall system resources
docker stats --no-stream

# Check disk usage
df -h
docker system df

# Monitor network usage
docker exec policy-engine netstat -i
docker exec fl-server ss -tuln

# Check memory and CPU limits
docker inspect policy-engine | jq '.[0].HostConfig.Memory'
docker inspect fl-server | jq '.[0].HostConfig.CpuShares'
\end{lstlisting}

\subsubsection{Database Performance Issues}

\begin{lstlisting}[style=bashcode, caption=Database Performance Debugging]
# Check SQLite database size and performance
docker exec policy-engine ls -lh /data/policies.db

# Analyze database queries (if SQLite3 available)
docker exec policy-engine sqlite3 /data/policies.db \
  ".timeout 2000" \
  "EXPLAIN QUERY PLAN SELECT * FROM policies;"

# Check database locks
docker exec policy-engine fuser /data/policies.db
\end{lstlisting}

\subsection{Configuration Issues}

\subsubsection{Environment Variable Problems}

\begin{lstlisting}[style=bashcode, caption=Environment Configuration Debugging]
# Check environment variables in running containers
docker exec policy-engine env | grep -E "(POLICY|FL|LOG)"
docker exec fl-server env | grep -E "(FL|POLICY|SERVER)"

# Verify Docker Compose environment loading
docker-compose config

# Check for environment file issues
cat .env | grep -v "^#" | grep -v "^$"
\end{lstlisting}

\subsubsection{Configuration File Validation}

\begin{lstlisting}[style=bashcode, caption=Configuration Validation]
# Validate JSON configuration files
find config/ -name "*.json" -exec echo "Checking {}" \; \
  -exec python -m json.tool {} > /dev/null \;

# Check Docker Compose file syntax
docker-compose config -q

# Validate policy configuration structure
docker exec policy-engine python -c "
import json
with open('/app/config/policies/policy_config.json') as f:
    config = json.load(f)
    print(f'Loaded {len(config.get(\"policies\", []))} policies')
"
\end{lstlisting}

\subsection{Logging and Monitoring}

\subsubsection{Log Aggregation and Analysis}

\begin{lstlisting}[style=bashcode, caption=Log Analysis Tools]
# Aggregate all service logs with timestamps
docker-compose logs -t > flopynet-logs.txt

# Filter logs by severity
docker-compose logs | grep -i error
docker-compose logs | grep -i warning

# Monitor real-time logs for specific patterns
docker-compose logs -f | grep -E "(ERROR|WARN|FAIL)"

# Extract performance metrics from logs
docker-compose logs collector | grep -E "(latency|throughput|response_time)"
\end{lstlisting}

\subsubsection{Health Check Implementation}

\begin{lstlisting}[style=bashcode, caption=Comprehensive Health Check Script]
#!/bin/bash
# health_check.sh - Comprehensive system health verification

echo "=== FLOPY-NET Health Check ==="

# Check Docker services
echo "Checking Docker services..."
docker-compose ps

# Check service health endpoints
services=("policy-engine:5000/health" "fl-server:8080/status" "collector:8000/health")

for service in "${services[@]}"; do
    echo "Checking $service..."
    if curl -f "http://localhost:${service#*:}" > /dev/null 2>&1; then
        echo "✓ $service is healthy"
    else
        echo "✗ $service is not responding"
    fi
done

# Check resource usage
echo "Checking resource usage..."
docker stats --no-stream --format "table {{.Container}}\t{{.CPUPerc}}\t{{.MemUsage}}"

# Check network connectivity
echo "Checking inter-service connectivity..."
docker exec policy-engine ping -c 1 fl-server > /dev/null 2>&1 && echo "✓ Policy Engine → FL Server" || echo "✗ Policy Engine → FL Server"

echo "Health check complete."
\end{lstlisting}

This troubleshooting guide provides systematic approaches to diagnosing and resolving issues across all FLOPY-NET components, enabling efficient problem resolution and system maintenance.
