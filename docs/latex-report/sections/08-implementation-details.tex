%============================================================================
% SECTION 8: IMPLEMENTATION DETAILS
%============================================================================
\section{Implementation Details}
\label{sec:implementation-details}

This section provides comprehensive technical implementation details of the FLOPY-NET platform, including technology stack choices, development practices, code organization, and key implementation patterns used throughout the system.

\subsection{Technology Stack Overview}

FLOPY-NET leverages a modern, cloud-native technology stack designed for scalability, maintainability, and research flexibility:

\begin{table}[H]
\centering
\caption{Technology Stack by Component}
\label{tab:technology-stack}
\begin{tabular}{@{}llp{6cm}@{}}
\toprule
\textbf{Component} & \textbf{Primary Technologies} & \textbf{Purpose \& Rationale} \\
\midrule
FL Framework & Python 3.8+, PyTorch \cite{pytorch}, Flower \cite{beutel2020flower} & Machine learning and federated learning capabilities \\
Policy Engine & Python, Flask, SQLite & Lightweight, fast policy enforcement \\
Dashboard Frontend & React 18, TypeScript, Material-UI & Modern, responsive user interface \\
Dashboard Backend & FastAPI, Python, asyncio & High-performance API server \\
Collector Service & Python, FastAPI, Redis, SQLite & High-throughput metrics collection \\
SDN Controller & Python, Ryu Framework \cite{ryu}, OpenFlow \cite{openflow2012specification} & Software-defined networking control \\
Network Simulation & GNS3, Docker, OpenVSwitch & Realistic network environment simulation \\
Orchestration & Docker, Docker Compose & Containerized deployment and scaling \\
\bottomrule
\end{tabular}
\end{table}

\subsection{Code Organization and Architecture}

The FLOPY-NET codebase follows a modular, service-oriented architecture with clear separation of concerns:

\begin{lstlisting}[style=pythoncode, caption=Project Structure Overview]
flopy-net/
├── src/                          # Core Python source code
│   ├── collector/                # Metrics collection service
│   ├── fl/                       # Federated learning framework
│   │   ├── server/               # FL server implementation
│   │   ├── client/               # FL client implementation
│   │   └── common/               # Shared FL utilities
│   ├── policy_engine/            # Policy enforcement service
│   ├── networking/               # Network simulation integration
│   └── utils/                    # Common utilities
├── dashboard/                    # Web dashboard application
│   ├── frontend/                 # React frontend
│   └── backend/                  # FastAPI backend
├── config/                       # Configuration files
├── docker/                       # Docker images and configurations
├── scripts/                      # Utility and deployment scripts
└── docs/                         # Documentation and reports

# Key implementation principles:
# 1. Each service is independently deployable
# 2. Configuration through environment variables and files
# 3. Comprehensive logging and monitoring
# 4. API-first design with OpenAPI specifications
# 5. Test-driven development with pytest
\end{lstlisting}

\subsection{Development Practices and Standards}

\subsubsection{Code Quality and Testing}

FLOPY-NET implements comprehensive quality assurance practices:

\begin{itemize}
    \item \textbf{Type Hints}: All Python code uses type hints for better maintainability
    \item \textbf{Linting}: Code quality enforced with pylint, black, and mypy
    \item \textbf{Testing}: Comprehensive test suite with pytest, including unit, integration, and end-to-end tests
    \item \textbf{Documentation}: Docstrings follow Google style, with automated API documentation generation
    \item \textbf{Version Control}: Git-based workflow with feature branches and pull request reviews
\end{itemize}

\subsubsection{Configuration Management}

The system implements a hierarchical configuration approach:

\begin{enumerate}
    \item Command-line arguments (highest priority)
    \item Environment variables
    \item Configuration files (JSON/YAML)
    \item Default values (lowest priority)
\end{enumerate}

\begin{lstlisting}[style=pythoncode, caption=Configuration Management Implementation]
import os
import json
from typing import Dict, Any, Optional
from dataclasses import dataclass, field

@dataclass
class ServiceConfig:
    """Base configuration class for all services."""
    host: str = field(default="0.0.0.0")
    port: int = field(default=8000)
    debug: bool = field(default=False)
    log_level: str = field(default="INFO")
    
    @classmethod
    def from_env(cls, prefix: str = "") -> 'ServiceConfig':
        """Load configuration from environment variables."""
        config_dict = {}
        
        for field_name, field_obj in cls.__dataclass_fields__.items():
            env_key = f"{prefix}{field_name.upper()}"
            env_value = os.getenv(env_key)
            
            if env_value is not None:
                # Type conversion based on field type
                if field_obj.type == bool:
                    config_dict[field_name] = env_value.lower() in ('true', '1', 'yes')
                elif field_obj.type == int:
                    config_dict[field_name] = int(env_value)
                else:
                    config_dict[field_name] = env_value
        
        return cls(**config_dict)
    
    @classmethod
    def from_file(cls, config_path: str) -> 'ServiceConfig':
        """Load configuration from JSON file."""
        with open(config_path, 'r') as f:
            config_dict = json.load(f)
        return cls(**config_dict)
    
    def merge_with_args(self, args: Dict[str, Any]) -> 'ServiceConfig':
        """Merge with command-line arguments."""
        config_dict = self.__dict__.copy()
        config_dict.update({k: v for k, v in args.items() if v is not None})
        return self.__class__(**config_dict)

# Usage example for Policy Engine
@dataclass
class PolicyEngineConfig(ServiceConfig):
    port: int = field(default=5000)
    policy_file: str = field(default="config/policies/policies.json")
    storage_backend: str = field(default="sqlite")
    cache_size: int = field(default=1000)
    
    @classmethod
    def load_config(cls) -> 'PolicyEngineConfig':
        """Load configuration using hierarchy."""
        # Start with defaults
        config = cls()
        
        # Override with file config if exists
        config_file = os.getenv("POLICY_CONFIG_FILE", "config/policy_engine.json")
        if os.path.exists(config_file):
            config = cls.from_file(config_file)
        
        # Override with environment variables
        env_config = cls.from_env("POLICY_")
        config = config.merge_with_args(env_config.__dict__)
        
        return config
\end{lstlisting}

% TODO: Continue with more implementation details including:
% - Service communication patterns
% - Error handling and resilience
% - Security implementation
% - Performance optimizations
% - Database design and migrations
% - API design patterns
% - Logging and monitoring integration

This section will be expanded with detailed implementation specifics for each component.
