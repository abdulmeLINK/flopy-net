%============================================================================
% SECTION 9: DEPLOYMENT ORCHESTRATION
%============================================================================
\section{Deployment Orchestration}
\label{sec:deployment-orchestration}

FLOPY-NET's deployment architecture leverages containerization and orchestration technologies to provide scalable, reproducible, and maintainable deployments across different environments. This section details the deployment strategies, container orchestration, and operational procedures.

\subsection{Container Architecture}

FLOPY-NET follows a microservices architecture where each component is containerized for independent deployment and scaling:

\begin{figure}[H]
\centering
\begin{tikzpicture}[
    node distance=2cm,    container/.style={rectangle, rounded corners, minimum width=2.5cm, minimum height=1.2cm, text centered, draw, thick, align=center},
    registry/.style={cylinder, minimum width=3cm, minimum height=1.5cm, text centered, draw, thick, fill=accent!20, align=center},
    flow/.style={->, thick, >=stealth}
]    % Docker Registry
    \node[registry] (registry) at (0,6) {Docker Hub\\ abdulmelink/*};
    
    % Container Images
    \node[container, fill=primary!20] (server) at (-6,3) {flopynet-server\\ FL Server};
    \node[container, fill=primary!20] (client) at (-3,3) {flopynet-client\\ FL Client};
    \node[container, fill=secondary!20] (policy) at (0,3) {flopynet-policy\\ Policy Engine};
    \node[container, fill=success!20] (collector) at (3,3) {flopynet-collector\\ Collector};
    \node[container, fill=accent!20] (controller) at (6,3) {flopynet-controller\\ SDN Controller};
    
    % Support Containers
    \node[container, fill=warning!20] (ovs) at (-4,1) {flopynet-openvswitch\\ Network Switch};
    \node[container, fill=warning!20] (dashboard) at (0,1) {flopynet-dashboard\\ Web Interface};
    \node[container, fill=warning!20] (gns3) at (4,1) {gns3/gns3server\\ Network Simulator};
    
    % Flows from registry
    \draw[flow, color=primary] (registry) -- (server);
    \draw[flow, color=primary] (registry) -- (client);
    \draw[flow, color=secondary] (registry) -- (policy);
    \draw[flow, color=success] (registry) -- (collector);
    \draw[flow, color=accent] (registry) -- (controller);
    \draw[flow, color=warning] (registry) -- (ovs);
    \draw[flow, color=warning] (registry) -- (dashboard);
\end{tikzpicture}
\caption{Container Architecture and Registry}
\label{fig:container-architecture}
\end{figure}

\subsection{GNS3 Test Environment Setup}

FLOPY-NET utilizes GNS3 as the network simulation backbone, providing realistic network topologies with FLOPY-NET Docker containers. The test environment consists of a GNS3 VM running the GNS3 server, which orchestrates network simulation and pulls FLOPY-NET containers from the abdulmelink registry.

\begin{figure}[H]
\centering
\begin{tikzpicture} [
    node distance=1.5cm,
    vm/.style={rectangle, rounded corners, minimum width=4cm, minimum height=6cm, text centered, draw, very thick, fill=primary!10, align=center},
    container/.style={rectangle, rounded corners, minimum width=2.8cm, minimum height=1cm, text centered, draw, thick, align=center},
    network/.style={rectangle, rounded corners, minimum width=2.8cm, minimum height=0.8cm, text centered, draw, thick, fill=accent!20, align=center},
    registry/.style={ellipse, minimum width=3cm, minimum height=2cm, text centered, draw, thick, fill=warning!20, align=center},
    flow/.style={->, thick, >=stealth},
    dashed_flow/.style={->, thick, >=stealth, dashed}
]
    % GNS3 VM Box
    \node[vm] (gns3_vm) at (0,0) {};
    \node[text width=3.5cm, align=center] at (0,2.5) {\textbf{GNS3 VM}\\ Ubuntu 20.04 LTS\\ 4 vCPUs, 8GB RAM\\ Docker Runtime};
    
    % GNS3 Server inside VM
    \node[container, fill=secondary!30] (gns3_server) at (0,1.5) {GNS3 Server\\ Port 3080};
    
    % FLOPY-NET Containers inside GNS3 VM
    \node[container, fill=primary!30] (fl_server_container) at (-1.2,0.5) {FL Server\\ :8080};
    \node[container, fill=primary!30] (fl_client_container) at (1.2,0.5) {FL Client\\ :8081};
    
    \node[container, fill=success!30] (policy_container) at (-1.2,-0.3) {Policy Engine\\ :5000};
    \node[container, fill=success!30] (collector_container) at (1.2,-0.3) {Collector\\ :8000};
    
    \node[container, fill=accent!30] (sdn_container) at (-1.2,-1.1) {SDN Controller\\ :6633};
    \node[container, fill=accent!30] (ovs_container) at (1.2,-1.1) {OpenVSwitch\\ :6640};
    
    % Virtual Networks
    \node[network] (fl_network) at (-3.5,0.5) {FL Network\\ 192.168.100.0/24};
    \node[network] (mgmt_network) at (3.5,0.5) {Management\\ 172.20.0.0/16};
    \node[network] (sdn_network) at (-3.5,-1.1) {SDN Network\\ 10.0.0.0/8};
    
    % External Docker Hub Registry
    \node[registry] (docker_hub) at (6,3) {Docker Hub\\ abdulmelink/*\\ Registry};
    
    % Host System
    \node[container, fill=warning!30, minimum width=5cm] (host_system) at (0,-3.5) {Host System\\ Windows/Linux\\ GNS3 Client + Docker};
    
    % Topology Templates
    \node[container, fill=info!30, minimum width=3cm] (templates) at (-6,1.5) {Custom Templates\\ FLOPY-NET Nodes};
    
    % Flows
    % Container pulls from registry
    \draw[flow, color=warning] (docker_hub) -- (gns3_server) node[midway, above right] {Pull Images};
    
    % Template injection
    \draw[dashed_flow, color=info] (templates) -- (gns3_server) node[midway, above] {Inject\\Templates};
    
    % Network connections
    \draw[flow, color=primary] (fl_network) -- (fl_server_container);
    \draw[flow, color=primary] (fl_network) -- (fl_client_container);
    \draw[flow, color=success] (mgmt_network) -- (policy_container);
    \draw[flow, color=success] (mgmt_network) -- (collector_container);
    \draw[flow, color=accent] (sdn_network) -- (sdn_container);
    \draw[flow, color=accent] (sdn_network) -- (ovs_container);
    
    % Host to VM connection
    \draw[flow, color=dark] (host_system) -- (gns3_vm) node[midway, right] {VM Control\\SSH/API};
    
    % Container orchestration within VM
    \draw[dashed_flow, color=secondary] (gns3_server) -- (fl_server_container);
    \draw[dashed_flow, color=secondary] (gns3_server) -- (fl_client_container);
    \draw[dashed_flow, color=secondary] (gns3_server) -- (policy_container);
    \draw[dashed_flow, color=secondary] (gns3_server) -- (collector_container);
    \draw[dashed_flow, color=secondary] (gns3_server) -- (sdn_container);
    \draw[dashed_flow, color=secondary] (gns3_server) -- (ovs_container);
\end{tikzpicture}
\caption{GNS3 Test Environment Architecture with FLOPY-NET Integration}
\label{fig:gns3-test-setup}
\end{figure}

\subsubsection{GNS3 Integration Components}

The GNS3 test environment consists of several key components:

\begin{table}[H]
\centering
\caption{GNS3 Test Environment Components}
\label{tab:gns3-components}
\begin{tabular}{@{}llp{6cm}@{}}
\toprule
\textbf{Component} & \textbf{Role} & \textbf{Description} \\
\midrule
GNS3 VM & Network Simulation Host & Ubuntu VM running GNS3 server with Docker runtime \\
GNS3 Server & Container Orchestrator & Manages network topology and container lifecycle \\
Docker Registry & Image Repository & abdulmelink/* images hosted on Docker Hub \\
Custom Templates & Node Definitions & FLOPY-NET component templates for GNS3 \\
Virtual Networks & Network Segmentation & Isolated networks for different traffic types \\
FLOPY-NET Containers & Simulation Components & Federated learning and network components \\
\bottomrule
\end{tabular}
\end{table}

\subsubsection{Container Deployment Process}

The deployment process follows these steps:

\begin{enumerate}
    \item \textbf{Template Injection}: Custom FLOPY-NET node templates are loaded into GNS3
    \item \textbf{Image Pulling}: GNS3 pulls the latest FLOPY-NET images from abdulmelik/* registry
    \item \textbf{Topology Creation}: Network topology is constructed using GNS3 GUI or API
    \item \textbf{Container Instantiation}: FLOPY-NET containers are deployed within the topology
    \item \textbf{Network Configuration}: Virtual networks and OpenVSwitch instances are configured
    \item \textbf{Service Startup}: All FLOPY-NET services are started in coordinated sequence
\end{enumerate}

\begin{figure}[H]
\centering
\begin{tikzpicture} [
    node distance=1.2cm,
    step/.style={rectangle, rounded corners, minimum width=3cm, minimum height=1cm, text centered, draw, thick, align=center},
    flow/.style={->, thick, >=stealth}
]
    \node[step, fill=primary!20] (templates) at (0,6) {Load Templates};
    \node[step, fill=secondary!20] (pull) at (0,4.5) {Pull Images};
    \node[step, fill=success!20] (topology) at (0,3) {Create Topology};
    \node[step, fill=accent!20] (deploy) at (0,1.5) {Deploy Containers};
    \node[step, fill=warning!20] (configure) at (0,0) {Configure Networks};
    \node[step, fill=info!20] (startup) at (0,-1.5) {Start Services};
    
    \draw[flow] (templates) -- (pull);
    \draw[flow] (pull) -- (topology);
    \draw[flow] (topology) -- (deploy);
    \draw[flow] (deploy) -- (configure);
    \draw[flow] (configure) -- (startup);
    
    % Side annotations
    \node[text width=4cm, right=1.5cm] at (templates.east) {Inject FLOPY-NET\\node definitions};
    \node[text width=4cm, right=1.5cm] at (pull.east) {Download latest\\container images};
    \node[text width=4cm, right=1.5cm] at (topology.east) {Design network\\architecture};
    \node[text width=4cm, right=1.5cm] at (deploy.east) {Instantiate\\containers};
    \node[text width=4cm, right=1.5cm] at (configure.east) {Setup virtual\\networks};
    \node[text width=4cm, right=1.5cm] at (startup.east) {Initialize\\FLOPY-NET};
\end{tikzpicture}
\caption{GNS3 Container Deployment Sequence}
\label{fig:gns3-deployment-sequence}
\end{figure}

\subsection{Docker Compose Configuration}

The system uses Docker Compose for orchestrating multi-container deployments:

\begin{lstlisting}[style=dockercode, caption=Main Docker Compose Configuration]
# docker-compose.yml
version: '3.8'

services:
  # Policy Engine - The Heart of the System
  policy-engine:
    image: abdulmelink/flopynet-policy:latest
    container_name: flopynet-policy-engine
    ports:
      - "5000:5000"
    environment:
      - POLICY_ENGINE_HOST=0.0.0.0
      - POLICY_ENGINE_PORT=5000
      - POLICY_CONFIG_FILE=/app/config/policies.json
      - LOG_LEVEL=INFO
    volumes:
      - ./config/policies:/app/config
      - ./logs:/app/logs
    networks:
      flopynet:
        ipv4_address: 172.20.0.5
    healthcheck:
      test: ["CMD", "curl", "-f", "http://localhost:5000/health"]
      interval: 30s
      timeout: 10s
      retries: 3
    restart: unless-stopped

  # Collector Service
  collector:
    image: abdulmelink/flopynet-collector:latest
    container_name: flopynet-collector
    ports:
      - "8000:8000"
    environment:
      - COLLECTOR_HOST=0.0.0.0
      - COLLECTOR_PORT=8000
      - REDIS_URL=redis://redis:6379
      - DATABASE_URL=sqlite:///data/metrics.db
    volumes:
      - ./data:/app/data
      - ./logs:/app/logs
    networks:
      flopynet:
        ipv4_address: 172.20.0.10
    depends_on:
      - policy-engine
      - redis
    restart: unless-stopped

  # FL Server
  fl-server:
    image: abdulmelink/flopynet-server:latest
    container_name: flopynet-fl-server
    ports:
      - "8080:8080"
      - "8081:8081"  # HTTP API
    environment:
      - FL_SERVER_HOST=0.0.0.0
      - FL_SERVER_PORT=8080
      - POLICY_ENGINE_URL=http://policy-engine:5000
      - COLLECTOR_URL=http://collector:8000
      - MIN_CLIENTS=2
      - MAX_CLIENTS=10
    volumes:
      - ./config/fl_server:/app/config
      - ./data:/app/data
    networks:
      flopynet:
        ipv4_address: 172.20.0.20
    depends_on:
      - policy-engine
      - collector
    restart: unless-stopped

  # FL Clients (can be scaled)
  fl-client-1:
    image: abdulmelink/flopynet-client:latest
    container_name: flopynet-fl-client-1
    environment:
      - CLIENT_ID=client_001
      - FL_SERVER_URL=http://fl-server:8080
      - POLICY_ENGINE_URL=http://policy-engine:5000
      - DATA_PATH=/app/data/client_1
    volumes:
      - ./data/clients/client_1:/app/data
      - ./config/fl_client:/app/config
    networks:
      flopynet:
        ipv4_address: 172.20.0.101
    depends_on:
      - fl-server
    restart: unless-stopped

  fl-client-2:
    image: abdulmelink/flopynet-client:latest
    container_name: flopynet-fl-client-2
    environment:
      - CLIENT_ID=client_002
      - FL_SERVER_URL=http://fl-server:8080
      - POLICY_ENGINE_URL=http://policy-engine:5000
      - DATA_PATH=/app/data/client_2
    volumes:
      - ./data/clients/client_2:/app/data
      - ./config/fl_client:/app/config
    networks:
      flopynet:
        ipv4_address: 172.20.0.102
    depends_on:
      - fl-server
    restart: unless-stopped

  # Dashboard Backend
  dashboard-backend:
    image: abdulmelink/flopynet-dashboard-backend:latest
    container_name: flopynet-dashboard-backend
    ports:
      - "8001:8001"
    environment:
      - DASHBOARD_HOST=0.0.0.0
      - DASHBOARD_PORT=8001
      - POLICY_ENGINE_URL=http://policy-engine:5000
      - COLLECTOR_URL=http://collector:8000
      - FL_SERVER_URL=http://fl-server:8080
    networks:
      flopynet:
        ipv4_address: 172.20.0.30
    depends_on:
      - policy-engine
      - collector
      - fl-server
    restart: unless-stopped

  # Dashboard Frontend
  dashboard-frontend:
    image: abdulmelink/flopynet-dashboard-frontend:latest
    container_name: flopynet-dashboard-frontend
    ports:
      - "8085:80"
    environment:
      - REACT_APP_API_URL=http://localhost:8001
    networks:
      flopynet:
        ipv4_address: 172.20.0.31
    depends_on:
      - dashboard-backend
    restart: unless-stopped

  # Redis for caching and message queuing
  redis:
    image: redis:7-alpine
    container_name: flopynet-redis
    ports:
      - "6379:6379"
    volumes:
      - redis_data:/data
    networks:
      flopynet:
        ipv4_address: 172.20.0.40
    restart: unless-stopped

  # GNS3 Server (External)
  # Note: This connects to external GNS3 server
  # Uncomment if running GNS3 in container
  # gns3-server:
  #   image: gns3/gns3server:latest
  #   container_name: flopynet-gns3-server
  #   ports:
  #     - "3080:3080"
  #   volumes:
  #     - gns3_projects:/opt/gns3-server/projects
  #     - /var/run/docker.sock:/var/run/docker.sock
  #   networks:
  #     flopynet:
  #       ipv4_address: 172.20.0.50
  #   restart: unless-stopped

networks:
  flopynet:
    driver: bridge
    ipam:
      config:
        - subnet: 172.20.0.0/16

volumes:
  redis_data:
  gns3_projects:
\end{lstlisting}

\subsection{Scaling and High Availability}

\subsubsection{Horizontal Scaling}

FLOPY-NET supports horizontal scaling for federated learning clients:

\begin{lstlisting}[style=dockercode, caption=Docker Compose Scaling]
# Scale FL clients dynamically
docker-compose up -d --scale fl-client=10

# Scale specific services
docker-compose up -d --scale collector=3 --scale dashboard-backend=2
\end{lstlisting}

\subsubsection{Load Balancing Configuration}

For production deployments, NGINX can be used as a load balancer:

\begin{lstlisting}[style=dockercode, caption=NGINX Load Balancer Configuration]
# nginx.conf for load balancing
upstream dashboard_backend {
    server dashboard-backend-1:8001;
    server dashboard-backend-2:8001;
    server dashboard-backend-3:8001;
}

upstream collector_service {
    server collector-1:8000;
    server collector-2:8000;
    server collector-3:8000;
}

server {
    listen 80;
    server_name flopynet.local;

    location /api/ {
        proxy_pass http://dashboard_backend;
        proxy_set_header Host $host;
        proxy_set_header X-Real-IP $remote_addr;
        proxy_set_header X-Forwarded-For $proxy_add_x_forwarded_for;
    }

    location /metrics/ {
        proxy_pass http://collector_service;
        proxy_set_header Host $host;
        proxy_set_header X-Real-IP $remote_addr;
    }

    location / {
        proxy_pass http://dashboard-frontend:80;
        proxy_set_header Host $host;
    }
}
\end{lstlisting}

\subsection{Environment Management}

\subsubsection{Environment Configuration}

Different deployment environments are managed through environment-specific configuration:

\begin{lstlisting}[style=dockercode, caption=Environment Configuration Files]
# .env.development
COMPOSE_PROJECT_NAME=flopynet-dev
ENVIRONMENT=development
LOG_LEVEL=DEBUG
POLICY_ENGINE_DEBUG=true
FL_SERVER_MIN_CLIENTS=2
GNS3_URL=http://localhost:3080

# .env.production
COMPOSE_PROJECT_NAME=flopynet-prod
ENVIRONMENT=production
LOG_LEVEL=INFO
POLICY_ENGINE_DEBUG=false
FL_SERVER_MIN_CLIENTS=5
GNS3_URL=http://gns3-server:3080
ENABLE_SSL=true

# .env.testing
COMPOSE_PROJECT_NAME=flopynet-test
ENVIRONMENT=testing
LOG_LEVEL=DEBUG
MOCK_NETWORK=true
FAST_TRAINING=true
\end{lstlisting}

\subsubsection{Configuration Templates}

The system uses configuration templates for different deployment scenarios:

\begin{table}[H]
\centering
\caption{Deployment Configuration Templates}
\label{tab:deployment-templates}
\begin{tabular}{@{}llp{6cm}@{}}
\toprule
\textbf{Template} & \textbf{Use Case} & \textbf{Configuration} \\
\midrule
Development & Local development & Single instance, debug enabled, fast startup \\
Testing & Automated testing & Mock components, deterministic behavior \\
Demo & Demonstrations & Lightweight, sample data, stable performance \\
Research & Research experiments & Full features, comprehensive logging \\
Production & Production deployment & High availability, security, monitoring \\
\bottomrule
\end{tabular}
\end{table}

\subsection{Deployment Automation}

\subsubsection{PowerShell Deployment Script}

Automated deployment script for Windows environments:

\begin{lstlisting}[style=dockercode, caption=PowerShell Deployment Script]
#!/usr/bin/env powershell
# deploy-flopynet.ps1

param(
    [Parameter(Mandatory=$false)]
    [ValidateSet("development", "testing", "production", "demo")]
    [string]$Environment = "development",
    
    [Parameter(Mandatory=$false)]
    [int]$ClientCount = 5,
    
    [Parameter(Mandatory=$false)]
    [switch]$Clean = $false,
    
    [Parameter(Mandatory=$false)]
    [switch]$Monitor = $false
)

$ErrorActionPreference = "Stop"

function Write-Status {
    param([string]$Message, [string]$Color = "Green")
    Write-Host "==> $Message" -ForegroundColor $Color
}

function Test-Prerequisites {
    Write-Status "Checking prerequisites..."
    
    # Check Docker
    try {
        docker --version | Out-Null
        docker-compose --version | Out-Null
    } catch {
        throw "Docker and Docker Compose are required"
    }
    
    # Check if ports are available
    $required_ports = @(5000, 8000, 8001, 8080, 8085)
    foreach ($port in $required_ports) {
        $connection = Test-NetConnection -ComputerName localhost -Port $port -WarningAction SilentlyContinue
        if ($connection.TcpTestSucceeded) {
            throw "Port $port is already in use"
        }
    }
    
    Write-Status "Prerequisites check passed"
}

function Initialize-Environment {
    Write-Status "Initializing $Environment environment..."
    
    # Copy environment-specific configuration
    if (Test-Path ".env.$Environment") {
        Copy-Item ".env.$Environment" ".env" -Force
        Write-Status "Loaded environment configuration: $Environment"
    }
    
    # Create required directories
    $directories = @("data", "logs", "data/clients")
    foreach ($dir in $directories) {
        if (-not (Test-Path $dir)) {
            New-Item -ItemType Directory -Path $dir -Force | Out-Null
        }
    }
    
    # Initialize client data directories
    for ($i = 1; $i -le $ClientCount; $i++) {
        $client_dir = "data/clients/client_$i"
        if (-not (Test-Path $client_dir)) {
            New-Item -ItemType Directory -Path $client_dir -Force | Out-Null
        }
    }
}

function Deploy-Services {
    Write-Status "Deploying FLOPY-NET services..."
    
    if ($Clean) {
        Write-Status "Cleaning previous deployment..."
        docker-compose down -v --remove-orphans
        docker system prune -f
    }
    
    # Pull latest images
    Write-Status "Pulling latest container images..."
    docker-compose pull
    
    # Start core services first
    Write-Status "Starting core services..."
    docker-compose up -d policy-engine redis
    
    # Wait for core services to be ready
    Start-Sleep 10
    
    # Start application services
    Write-Status "Starting application services..."
    docker-compose up -d collector fl-server
    
    # Start clients
    Write-Status "Starting FL clients (count: $ClientCount)..."
    docker-compose up -d --scale fl-client=$ClientCount
    
    # Start dashboard
    Write-Status "Starting dashboard..."
    docker-compose up -d dashboard-backend dashboard-frontend
    
    Write-Status "Deployment completed successfully!"
}

function Test-Deployment {
    Write-Status "Testing deployment..."
    
    $services = @(
        @{Name="Policy Engine"; URL="http://localhost:5000/health"},
        @{Name="Collector"; URL="http://localhost:8000/health"},
        @{Name="FL Server"; URL="http://localhost:8080/health"},
        @{Name="Dashboard API"; URL="http://localhost:8001/health"},
        @{Name="Dashboard"; URL="http://localhost:8085"}
    )
    
    foreach ($service in $services) {
        try {
            $response = Invoke-RestMethod -Uri $service.URL -TimeoutSec 10
            Write-Status "$($service.Name): OK" "Green"
        } catch {
            Write-Status "$($service.Name): FAIL" "Red"
        }
    }
}

function Show-Status {
    Write-Status "FLOPY-NET Deployment Status"
    Write-Host "============================" -ForegroundColor Cyan
    
    docker-compose ps
    
    Write-Host ""
    Write-Host "Access URLs:" -ForegroundColor Cyan
    Write-Host "  Dashboard: http://localhost:8085" -ForegroundColor White
    Write-Host "  API Docs:  http://localhost:8001/docs" -ForegroundColor White
    Write-Host "  Policy Engine: http://localhost:5000" -ForegroundColor White
    Write-Host "  Collector: http://localhost:8000" -ForegroundColor White
    
    if ($Monitor) {
        Write-Status "Starting monitoring (Ctrl+C to exit)..."
        docker-compose logs -f
    }
}

# Main execution
try {
    Write-Status "FLOPY-NET Deployment Script" "Cyan"
    Write-Host "Environment: $Environment" -ForegroundColor Yellow
    Write-Host "Client Count: $ClientCount" -ForegroundColor Yellow
    
    Test-Prerequisites
    Initialize-Environment
    Deploy-Services
    Start-Sleep 5
    Test-Deployment
    Show-Status
    
} catch {
    Write-Host "Deployment failed: $_" -ForegroundColor Red
    exit 1
}
\end{lstlisting}

\subsection{Health Monitoring and Maintenance}

\subsubsection{Health Checks}

Each service implements comprehensive health checks:

\begin{lstlisting}[style=pythoncode, caption=Service Health Check Implementation]
from fastapi import FastAPI, HTTPException
from typing import Dict, Any
import psutil
import time
import asyncio

class HealthChecker:
    def __init__(self, service_name: str):
        self.service_name = service_name
        self.start_time = time.time()
        self.dependencies = []
    
    async def check_health(self) -> Dict[str, Any]:
        """Comprehensive health check."""
        try:
            health_status = {
                "service": self.service_name,
                "status": "healthy",
                "timestamp": time.time(),
                "uptime": time.time() - self.start_time,
                "version": "2.0.0",
                "system": await self._check_system_health(),
                "dependencies": await self._check_dependencies()
            }
            
            # Determine overall status
            if any(dep["status"] != "healthy" for dep in health_status["dependencies"]):
                health_status["status"] = "degraded"
            
            return health_status
            
        except Exception as e:
            return {
                "service": self.service_name,
                "status": "unhealthy",
                "error": str(e),
                "timestamp": time.time()
            }
    
    async def _check_system_health(self) -> Dict[str, Any]:
        """Check system-level health metrics."""
        try:
            cpu_percent = psutil.cpu_percent(interval=1)
            memory = psutil.virtual_memory()
            disk = psutil.disk_usage('/')
            
            return {
                "cpu_usage": cpu_percent,
                "memory_usage": {
                    "total": memory.total,
                    "used": memory.used,
                    "percentage": memory.percent
                },
                "disk_usage": {
                    "total": disk.total,
                    "used": disk.used,
                    "percentage": (disk.used / disk.total) * 100
                }
            }
        except Exception as e:
            return {"error": str(e)}
    
    async def _check_dependencies(self) -> List[Dict[str, Any]]:
        """Check health of dependent services."""
        dependency_results = []
        
        for dep in self.dependencies:
            try:
                # Attempt to connect to dependency
                result = await self._ping_service(dep["url"])
                dependency_results.append({
                    "name": dep["name"],
                    "status": "healthy" if result else "unhealthy",
                    "url": dep["url"],
                    "response_time": result.get("response_time") if result else None
                })
            except Exception as e:
                dependency_results.append({
                    "name": dep["name"],
                    "status": "error",
                    "error": str(e)
                })
        
        return dependency_results
\end{lstlisting}

The deployment orchestration framework provides FLOPY-NET with robust, scalable, and maintainable deployment capabilities across different environments and use cases.
